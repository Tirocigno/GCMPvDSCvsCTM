%%%%%%%%%%%%%%%%%%%%%%%%%%%%%%%%%%%%%%%%%
% Beamer Presentation
% LaTeX Template
% Version 1.0 (10/11/12)
%
% This template has been downloaded from:
% http://www.LaTeXTemplates.com
%
% License:
% CC BY-NC-SA 3.0 (http://creativecommons.org/licenses/by-nc-sa/3.0/)
%
%%%%%%%%%%%%%%%%%%%%%%%%%%%%%%%%%%%%%%%%%

%----------------------------------------------------------------------------------------
%	PACKAGES AND THEMES
%----------------------------------------------------------------------------------------

\documentclass{beamer}

\mode<presentation> {

% The Beamer class comes with a number of default slide themes
% which change the colors and layouts of slides. Below this is a list
% of all the themes, uncomment each in turn to see what they look like.

%\usetheme{default}
%\usetheme{AnnArbor}
%\usetheme{Antibes}
%\usetheme{Bergen}
%\usetheme{Berkeley}
%\usetheme{Berlin}
%\usetheme{Boadilla}
%\usetheme{CambridgeUS}
%\usetheme{Copenhagen}
%\usetheme{Darmstadt}
%\usetheme{Dresden}
%\usetheme{Frankfurt}
%\usetheme{Goettingen}
%\usetheme{Hannover}
%\usetheme{Ilmenau}
%\usetheme{JuanLesPins}
%\usetheme{Luebeck}
\usetheme{Madrid}
%\usetheme{Malmoe}
%\usetheme{Marburg}
%\usetheme{Montpellier}
%\usetheme{PaloAlto}
%\usetheme{Pittsburgh}
%\usetheme{Rochester}
%\usetheme{Singapore}
%\usetheme{Szeged}
%\usetheme{Warsaw}

% As well as themes, the Beamer class has a number of color themes
% for any slide theme. Uncomment each of these in turn to see how it
% changes the colors of your current slide theme.

%\usecolortheme{albatross}
%\usecolortheme{beaver}
%\usecolortheme{beetle}
%\usecolortheme{crane}
%\usecolortheme{dolphin}
%\usecolortheme{dove}
%\usecolortheme{fly}
%\usecolortheme{lily}
%\usecolortheme{orchid}
%\usecolortheme{rose}
%\usecolortheme{seagull}
%\usecolortheme{seahorse}
%\usecolortheme{whale}
%\usecolortheme{wolverine}

%\setbeamertemplate{footline} % To remove the footer line in all slides uncomment this line
%\setbeamertemplate{footline}[page number] % To replace the footer line in all slides with a simple slide count uncomment this line

%\setbeamertemplate{navigation symbols}{} % To remove the navigation symbols from the bottom of all slides uncomment this line
}

\usepackage{graphicx} % Allows including images
\usepackage{booktabs} % Allows the use of \toprule, \midrule and \bottomrule in tables
\graphicspath{ {../img/} }%setta il path predefinito per le immagini
\usepackage[utf8]{inputenc}
\usepackage{hyperref}
\setlength{\columnseprule}{0.4pt}
\usepackage{multicol}

%----------------------------------------------------------------------------------------
%	TITLE PAGE
%----------------------------------------------------------------------------------------

\title[Trajectory Clustering Comparison]{Trajectory Clustering Alghorithms - GCMP vs DSC vs CTM. } % The short title appears at the bottom of every slide, the full title is only on the title page

\author{Federico Naldini} % Your name
\institute[Università di Bologna] % Your institution as it will appear on the bottom of every slide, may be shorthand to save space
{
Alma Mater Studiorum - Università di Bologna, Cesena. \\ % Your institution for the title page
\medskip
\textit{federico.naldini3@studio.unibo.it} % Your email address
}
\date{18/10/2019} % Date, can be changed to a custom date

\begin{document}

\begin{frame}
\titlepage % Print the title page as the first slide
\end{frame}

%------------------------------------------------

\begin{frame}
	\frametitle{CTM: Overview}
	Framework per l'identificazione di gruppi di oggetti che hanno viaggiato assieme per un certo periodo di tempo.
	
	Si basa sulla divisione dell'area di viaggio degli oggetti in una griglia di celle, per ogni cella vengono raggruppate le traiettorie che sono passate da lì.
	
	A questo punto le celle vengono identificate come le transazioni, le traiettorie come item e tramite l'algoritmo \textit{Apriori} si generano e si restituiscono in output tutti i \textit{frequent closed itemset}.
\end{frame}

\begin{frame}
	\frametitle{Comparison: Overview}
	\begin{columns}
		
		\column{.3\columnwidth}
		\begin{center}
			\textbf{\textit{\huge{GCMP}}}
			
		\end{center}
		
		\column{.3\columnwidth}
		\begin{center}
			\textbf{\textit{\huge{DSC}}}
			
		\end{center}
		
		\column{.3\columnwidth}
		\begin{center}
			\textbf{\textit{\huge{CTM}}}
			
		\end{center}
	\end{columns}
	\begin{columns}
		
		\column{.3\columnwidth}
		
	Framework dedicato per il riconoscimento di \textit{Co-movement patterns} in maniera distribuita.
		
		
		\column{.3\textwidth}
	
			Framework che, dato un insieme di traiettorie, riconosce e clusterizza le \textit{sub-trajectories} estratte da queste.
		
		\column{.3\textwidth}
		
		Basandosi su \textit{frequent itemset mining}, individua oggetti  che hanno viaggiato assieme per un insieme di istanti non continui.

	\end{columns}
\end{frame}     

\begin{frame}
		\frametitle{Comparison: criterio spaziale}
		\begin{columns}
			
			\column{.3\columnwidth}
			\begin{center}
				\textbf{\textit{\huge{GCMP}}}
				
			\end{center}
		
			\column{.3\columnwidth}
		\begin{center}
			\textbf{\textit{\huge{DSC}}}
			
		\end{center}
	
		\column{.3\columnwidth}
	\begin{center}
		\textbf{\textit{\huge{CTM}}}
		
	\end{center}
		\end{columns}
		\begin{columns}
			
			\column{.3\columnwidth}
			
			Utilizza un algoritmo di clustering 
			\textit{density-based} o \textit{distance-based}.
		
			
				\column{.3\textwidth}
		
			Impiega una variante pesata di \textit{LCSS} che definisce un range spazio-temporale.
			
				\column{.3\textwidth}
			    
				Utilizza un criterio di raggruppamento basato sulla divisione dell'area
				in cui si muovono gli oggetti in celle.
		\end{columns}
	\end{frame}     

\begin{frame}
	\frametitle{Comparison: criterio temporale}
	\begin{columns}
		
		\column{.3\columnwidth}
		\begin{center}
			\textbf{\textit{\huge{GCMP}}}
			
		\end{center}
		
		\column{.3\columnwidth}
		\begin{center}
			\textbf{\textit{\huge{DSC}}}
			
		\end{center}
		
		\column{.3\columnwidth}
		\begin{center}
			\textbf{\textit{\huge{CTM}}}
			
		\end{center}
	\end{columns}
	\begin{columns}
		
		\column{.3\columnwidth}
		
		\textit{Apriori Enumeration}: consente il pruning degli insiemi di oggetti
		che non rispettano i criteri di \textit{L-Consecutivness} e \textit{G-connection}.
		
		
		\column{.3\textwidth}
		
		Impiega una variante pesata di \textit{LCSS} che definisce un range spazio-temporale.
		
		\column{.3\textwidth}
		Nativamente ignorato.
		
		Tuttavia è possibile aggiungere una dimensione temporale alle celle, rendendole di fatto cubi.
		
	\end{columns}
\end{frame}    

\begin{frame}
	\frametitle{Comparison: parametri in gioco}
	\begin{columns}
		
		\column{.3\columnwidth}
		\begin{center}
			\textbf{\textit{\huge{GCMP}}}
			
		\end{center}
		
		\column{.3\columnwidth}
		\begin{center}
			\textbf{\textit{\huge{DSC}}}
			
		\end{center}
		
		\column{.3\columnwidth}
		\begin{center}
			\textbf{\textit{\huge{CTM}}}
			
		\end{center}
	\end{columns}
	\begin{columns}
		
		\column{.3\columnwidth}

\begin{itemize}
	\item \textbf{M}: numero minimo di elementi.
	\item \textbf{K}: numero minimo di istanti.
	\item \textbf{L}: lunghezza minima sottosequenze consecutive.
	\item \textbf{G}: massimo intervallo tra un istante e il successivo.
\end{itemize}
		
		
		\column{.3\textwidth}
		
			\begin{itemize}
			\item \textbf{$\epsilon$\textsubscript{sp}}: tolleranza spaziale.
			\item \textbf{$\epsilon$\textsubscript{t}}: tolleranza temporale.
			\item \textbf{K}: limite inferiore al voting per rappresentante.
			\item \textbf{$\alpha$}: soglia di coesione per i cluster. 
		\end{itemize}
		
		\column{.3\textwidth}
		
	\begin{itemize}
		\item \textbf{MinSize}: numero minimo di elementi per un raggruppamento.
		\item \textbf{MinSup}: minimo di percorsi assieme in x celle.
		\item \textbf{MinCoh}: minimo coesione per un itemset.
	\end{itemize}
		
	\end{columns}
\end{frame}    

\begin{frame}
	\frametitle{Comparison: preprocessing}
	\begin{columns}
		
		\column{.3\columnwidth}
		\begin{center}
			\textbf{\textit{\huge{GCMP}}}
			
		\end{center}
		
		\column{.3\columnwidth}
		\begin{center}
			\textbf{\textit{\huge{DSC}}}
			
		\end{center}
		
		\column{.3\columnwidth}
		\begin{center}
			\textbf{\textit{\huge{CTM}}}
			
		\end{center}
	\end{columns}
	\begin{columns}
		
		\column{.3\columnwidth}
		
		Unificazione dei campionamenti temporali in scala  (una volta sola per dataset).
		
		
		\column{.3\textwidth}
		
		Unificazione dei campionamenti temporali in scala.
		
		Costruzione istogramma \textit{equi-depth} sul tempo e partizionamento dei dati in \textit{buckets}
		basati su questo  (una volta sola per dataset).
		
		\column{.3\textwidth}
		
		Calcolo dell'area in cui si muovono gli oggetti e generazione del reticolo di celle.
		
		Eventuale generazione di una misura univoca per il tempo e divisione in intervalli 
		 (una volta sola per dataset).
		
	\end{columns}
\end{frame}  



\begin{frame}
	\frametitle{Comparison: GCMP and CTM = Swarm}
 \textit{Co-Movement} pattern in cui i vincoli temporali sono praticamente assenti, 
 rimangono solamente quelli spaziali che possono essere mappati come segue:
 
 \begin{center}
 	\textbf{M}\textless ---\textgreater  \textbf{minSize}
 	
 	\textbf{eps} \textless ---\textgreater  \textbf{minCoh} (valido solo in locale)
 	
 	
 \end{center}
Rilassando ogni vincolo temporale al massimo, il risultato ottenuto dalla ricerca di un pattern swarm può avvicinarsi molto all'output di \textit{CTM}.
\end{frame}

\begin{frame}
	\frametitle{Comparison: DSC and CTM }
E' impossibile rilasciare completamente vincolo temporale su DSC, data l'implementazione dell'algoritmo e il parizionamento dei dati in bucket di stessa densità.

Si può provare a impostare una dimensione temporale sulle celle di \textit{CTM} coincidente con l'istogramma individuato per DSC, inoltre per determinare le sottosequenze si può provare a impostare come criterio di partizionamento delle traiettorie un cambio del vicinato.

Tuttavia DSC rimane molto più preciso di CTM e molti itemset individuati da CTM sarebbero invece scartati da DSC(vedi esempio) .
\end{frame}

\begin{frame}
	\frametitle{Comparison: Limits of each framework}
	\begin{columns}
		
		\column{.3\columnwidth}
		\begin{center}
			\textbf{\textit{\huge{GCMP}}}
			
		\end{center}
		
		\column{.3\columnwidth}
		\begin{center}
			\textbf{\textit{\huge{DSC}}}
			
		\end{center}
		
		\column{.3\columnwidth}
		\begin{center}
			\textbf{\textit{\huge{CTM}}}
			
		\end{center}
	\end{columns}
	\begin{columns}
		
		\column{.3\columnwidth}
		
\begin{itemize}
	\item Assenza di pruning sulla base del principio Apriori su criterio spaziale.
	\item Coesione solamente locale agli snapshot, globalmente viene utilizzato
	un \textit{density connected criteria}.
\end{itemize}
		
		
		\column{.3\textwidth}
\begin{itemize}
	\item Poca flessibilità.
	\item L'idea di fondo è differente.
\end{itemize}

		
		\column{.3\textwidth}
\begin{itemize}
	\item Nessun supporto alla continuità negli istanti temporali.
	\item Molto legato alla divisione della mappa in celle, raggruppamenti che stanno sui bordi potrebbero non essere riconosciuti.
\end{itemize}
	\end{columns}
\end{frame}     



\end{document}
